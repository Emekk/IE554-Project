\documentclass[12pt]{article}
\usepackage[utf8]{inputenc}
\usepackage{enumitem}
\usepackage{graphicx}
\usepackage[autostyle]{csquotes}
\usepackage{amsmath}
\usepackage[T1]{fontenc}
\usepackage{romannum}
\usepackage{calligra}
\usepackage{newtxmath}
\usepackage{float}
\usepackage{pifont}
\usepackage{setspace}
\usepackage{subcaption}
\usepackage[export]{adjustbox}
\usepackage{tcolorbox}
\usepackage{multicol}
\usepackage[bottom]{footmisc}
\usepackage{hyperref}
\usepackage[a4paper, margin=2cm]{geometry}
\usepackage[all]{hypcap}

\hypersetup{
    colorlinks=true,
    linktoc=all,
    linkcolor=black,
    citecolor=black,
}

\renewcommand{\ttdefault}{cmvtt}
\renewcommand\contentsname{CONTENTS}
\setcounter{tocdepth}{3}
\setstretch{1.5}
\renewcommand{\arraystretch}{1.5}
\setlength{\skip\footins}{1.5cm}
\setlength{\footnotesep}{0.6cm}
\makeatletter
\renewcommand\@makefntext[1]{\noindent\makebox[12pt][r]{\@thefnmark. }#1}

\begin{document}

\begin{tcolorbox}[colback=white, bottomrule=1mm, sharp corners]
    2024-2025 Spring
    \begin{center}
        \Large
        \textbf{IE 554 Project -- Mathematical Models}
        \\[4mm]
        \normalsize
        \hfill \textsl{Emek Irmak \& Ömer Turan Şahinaslan}
    \end{center}
\end{tcolorbox}

\pagenumbering{gobble}
\tableofcontents
\clearpage
\pagenumbering{arabic}

\section{Introduction}
Graph domination problems play a central role in combinatorial optimization and graph theory. Two recently introduced variants -- the \textbf{Dominator Partition Problem} and the \textbf{Dominator Coloring Problem} -- focus on domination-based graph structures. In this project, we propose exact \textbf{integer programming formulations} for both problems, aiming to strengthen them with valid inequalities and analyze their computational performance.

\section{Problem Definitions}

\subsection{Dominator Partition Problem}
Given a graph $G = (V, E)$, a vertex $v \in V$ \textit{dominates} a set $S \subseteq V$ if $v$ is adjacent to every vertex in $S$. A \textbf{dominator partition} $\pi = \{V_1, V_2, \dots, V_k\}$ of $V$ satisfies that each vertex dominates at least one block $V_i$. The \textbf{dominator partition number} $\pi_d(G)$ is the minimum number $k$ such that such a partition exists.

\subsection{Dominator Coloring Problem}
Given a graph $G = (V, E)$, a \textbf{dominator coloring} is a proper coloring such that each vertex dominates at least one color class. The objective is to minimize the number of colors used, known as the \textbf{dominator chromatic number} $\chi_d(G)$.

\section{Literature Review}

While dominator partitions and dominator colorings have been studied theoretically \cite{dominator_partitions}, focusing on bounds, properties, and complexity, there has been \textbf{no prior integer programming formulation} for either problem. It is known that deciding whether a graph admits a dominator partition or dominator coloring with bounded size is NP-complete. However, optimization-based approaches have not been explored in the literature.

Thus, this project addresses a novel research gap by developing IP models for both problems and improving their computational efficiency.

\section{Mathematical Models}

\subsection{Model for Dominator Partition Problem}
Let us define:
\begin{itemize}[label=, noitemsep]
    \item $V$: set of vertices.
    \item $a_{vu}$: 1 if $v$ and $u$ are adjacent.
\end{itemize}

\textbf{Decision Variables:}
\begin{itemize}[label=, noitemsep]
    \item $x_{vi}$: 1 if vertex $v$ belongs to block $i$.
    \item $y_i$: 1 if block $i$ is non-empty.
\end{itemize}

\textbf{Objective and Constraints:}
\begin{align*}
    \min \quad & \sum_{i=1}^{n} y_i \\
    \text{s.t.} \quad
    & \sum_{i=1}^{n} x_{vi} = 1 \quad \forall v \in V \\
    & x_{ui} \leq a_{vu} + (1 - x_{vi}) \quad \forall v \in V, u \in V, i=1,\dots,n \\
    & x_{vi} \leq y_i \quad \forall v \in V, i=1,\dots,n \\
    & x_{vi}, y_i \in \{0,1\} \quad \forall v \in V, i=1,\dots,n.
\end{align*}

\subsection{Model for Dominator Coloring Problem}
Sets and parameters are the same.

\textbf{Decision Variables:}
\begin{itemize}[label=, noitemsep]
    \item $x_{vi}$: 1 if vertex $v$ is assigned color $i$.
    \item $y_i$: 1 if color $i$ is used.
    \item $d_{vi}$: 1 if vertex $v$ dominates color class $i$.
\end{itemize}

\textbf{Objective and Constraints:}
\begin{align*}
    \min \quad & \sum_{i=1}^{n} y_i \\
    \text{s.t.} \quad
    & \sum_{i=1}^{n} x_{vi} = 1 \quad \forall v \in V \\
    & x_{ui} + x_{vi} \leq 2 - a_{uv} \quad \forall u \neq v \in V, i=1,\dots,n \\
    & x_{vi} \leq y_i \quad \forall v \in V, i=1,\dots,n \\
    & x_{ui} \leq a_{vu} + (1 - d_{vi}) \quad \forall v, u \in V, u \neq v, i=1,\dots,n \\
    & d_{vi} \leq y_i \quad \forall v \in V, i=1,\dots,n \\
    & \sum_{i=1}^{n} d_{vi} \geq 1 \quad \forall v \in V \\
    & x_{vi}, y_i, d_{vi} \in \{0,1\} \quad \forall v \in V, i=1,\dots,n.
\end{align*}

\section{Planned Contributions}
\begin{itemize}[leftmargin=*, noitemsep]
    \item \textbf{Exact Models}: This will be the first known IP models for dominator partition and dominator coloring.
    \item \textbf{Model Strengthening}: We aim to derive valid inequalities to strengthen the above formulation.
    \item \textbf{Computational Analysis}: We aim to compare models' performance on graphs of various sizes.
    \item \textbf{Alternative Approaches}: We aim to explore possible new formulations for both problems.
\end{itemize}

\section{Conclusion}

By formulating both of the dominator-based graph problems  -- the \textbf{Dominator Partition Problem} and the \textbf{Dominator Coloring Problem} -- as integer programs and improving their solvebility through strengthening and computational testing, we aim to significantly contribute to the understanding and solution of these problems.

\clearpage

\bibliographystyle{plain}
\begin{thebibliography}{9}

\bibitem{dominator_partitions}
S. M. Hedetniemi, S. T. Hedetniemi, R. Laskar, A. A. McRae, C. K. Wallis,
\textit{Dominator Partitions of Graphs}, 2008.

\end{thebibliography}

\end{document}
