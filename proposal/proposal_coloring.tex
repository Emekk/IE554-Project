\documentclass[12pt]{article}
\usepackage[utf8]{inputenc}
\usepackage{enumitem}
\usepackage{graphicx}
\usepackage[autostyle]{csquotes}
\usepackage{amsmath}
\usepackage[T1]{fontenc}
\usepackage{romannum}
\usepackage{calligra} % handwritten font
%\usepackage{newtxtext} % For text font
\usepackage{newtxmath}
\usepackage{float}
%\usepackage{pdfpages}
\usepackage{pifont}
\usepackage{setspace}
\usepackage{subcaption}
\usepackage[export]{adjustbox}
\usepackage{tcolorbox}
\usepackage{multicol}
%\usepackage{pgfplots}  % matplotlib plots
%\usepackage{tikz}
%\usetikzlibrary{shapes.geometric, shapes.multipart, arrows, positioning, arrows.meta}
\usepackage[bottom]{footmisc}  % fix footnotes at the bottom of the page
\usepackage{hyperref}
\usepackage[a4paper, margin=1cm]{geometry}
%\usepackage[toc, section=section]{glossaries}
\usepackage[all]{hypcap}  % needed to help hyperlinks direct correctly

% hyperlinks
\hypersetup{
    colorlinks=true, %set true if you want colored links
    linktoc=all,     %set to all if you want both sections and subsections linked
    linkcolor=black, %choose some color if you want links to stand out
}

% typewriter font
\renewcommand{\ttdefault}{cmvtt}

% table of contents
\renewcommand\contentsname{CONTENTS}
\setcounter{tocdepth}{3}

% custom section commands
%\newcommand\subsectiontoc[1]{
%    \subsection*{\underline{#1}}
%    \addcontentsline{toc}{subsection}{#1}
%    \markboth{#1}{#1}}
%\newcommand\subsubsectiontoc[1]{
%    \subsubsection*{\underline{#1}}
%    \addcontentsline{toc}{subsubsection}{#1}
%    \markboth{#1}{#1}}

% letter subsections
%\renewcommand{\thesubsection}{\thesection.\alph{subsection}}

% line spacing
\setstretch{1.5}
\renewcommand{\arraystretch}{1.5} % table row spacing

% footnote
\setlength{\skip\footins}{1.5cm}
\setlength{\footnotesep}{0.6cm}
\makeatletter
\renewcommand\@makefntext[1]{\noindent\makebox[12pt][r]{\@thefnmark. }#1}

% glossary
%\input{glossary/glossary.tex}

% tikz styles
%\tikzstyle{startstop} = [rectangle, rounded corners=4mm, minimum width=3cm, minimum height=1cm, text centered, draw=black, fill=white]
%\tikzstyle{io} = [trapezium, trapezium left angle=70, trapezium right angle=110, minimum width=3cm, minimum height=1cm, text centered, draw=black, fill=white]
%\tikzstyle{process} = [rectangle, minimum width=3cm, minimum height=1cm, text centered, draw=black, fill=white]
%\tikzstyle{decision} = [diamond, minimum width=3cm, minimum height=1cm, aspect=1.75, text centered, draw=black, fill=white]
%\tikzstyle{arrow} = [thick, ->, >=latex]


\begin{document}

\begin{tcolorbox}[colback=white, bottomrule=1mm, sharp corners]
    2024-2025 Spring
    \begin{center}
        \Large
        \textbf{IE 554 Project -- Mathematical Models}
        \\[4mm]
        \normalsize
        \hfill \textsl{Emek Irmak \& Ömer Turan Şahinaslan}
    \end{center}
\end{tcolorbox}

\pagenumbering{gobble}
\tableofcontents
\clearpage
\pagenumbering{arabic}

\section{Introduction}

Graph partitioning problems are fundamental in combinatorial optimization and graph theory, offering a wide range of applications from network design to clustering. In this project, we focus on a novel variant called the \textbf{Dominator Partition Problem}, recently introduced in the literature by Hedetniemi and Haynes (2006) \cite{dominator_partitions}. We propose integer programming formulation for the problem and try to improve it. 

\section{Problem Definition}

Given a graph $G = (V, E)$, a vertex $v \in V$ \textit{dominates} a set $S \subseteq V$ if $v$ is adjacent to every vertex in $S$. A \textbf{dominator partition} $\pi = \{V_1, V_2, \dots, V_k\}$ of $V$ is a partition such that each vertex $v \in V$ dominates at least one block $V_i$ in the partition. The \textit{dominator partition number} $\pi_d(G)$ is the minimum number $k$ such that $G$ has a dominator partition of order $k$, and our goal is to find it.


For this project, we propose an \textbf{integer programming model} to solve the dominator partition problem exactly. Then, we try to improve the formulation such as making the model stronger such that it can be solved faster on bigger graphs.

The study process goes like this: first we tried to model the problem mathematically using integer programming. Then we plan to work on improving the model, mainly by adding valid inequalities. Maybe later we will find other ways to solve the problem and compare those methods with our original model. Also, after fixing the value of $k$, we want to try different formulations and compare their runtime performances.

\section{Literature Review}

Since dominator partitions were introduced, most studies have been about their properties, complexity and special graph classes like trees and cycles \cite{dominator_partitions}. It was shown that checking if a graph has a dominator partition with at most $k$ blocks is NP-complete, even when the graph is bipartite, planar, or chordal.

However, \textbf{no one has formulated the dominator partition problem using an integer programming model before}, to our knowledge. The existing work is mostly theoretical, without optimization models. So, our study brings a new angle by using integer programming to find dominator partitions.


\noindent
\begin{minipage}[t]{0.48\textwidth}
\textbf{Sets \& Parameters}
\begin{itemize}[label=, noitemsep, topsep=0pt, leftmargin=2mm]
    \item $V$: set of vertices in the graph, of size $n$
    \item $a_{vu}$: 1 if vertices $v$ and $u$ are adjacent
    \item $k$: number of blocks in the partition
\end{itemize}
\end{minipage}
\hfill
\begin{minipage}[t]{0.48\textwidth}
\textbf{Decision Variables}
\begin{itemize}[label=, noitemsep, topsep=0pt, leftmargin=2mm]
    \item $x_{vi}$: 1 if vertex $v$ is assigned to the $i$\textsuperscript{th} block
    \item $d_{vi}$: 1 if vertex $v$ dominates block $i$
\end{itemize}
\end{minipage}

\subsection*{Objective Function \& Constraints}
\begin{align*}
    \min \quad &0 &&\text{(no objective)}\\
    \text{s.t.} \quad
    &\sum_{i=1}^{k} x_{vi} = 1, \quad \forall v \in V &&\text{(each vertex is assigned to one block)}\\
    &\sum_{v \in V} x_{vi} \geq 1 \quad \forall i \in \{1, 2, \dots, k\} &&\text{(no empty blocks)}\\
    &x_{ui} \leq a_{vu} + (1 - d_{vi}) \quad \forall u,v \in V, i \in \{1, 2, \dots, k\} &&\text{(domination condition)}\\
    &\sum_{i=1}^{k} d_{vi} \geq 1 \quad \forall v \in V &&\text{(each vertex dominates at least one block)}\\
    & \sum_{v \in V} x_{vi} \geq \sum_{v \in V} x_{v,i+1} \quad \forall i \in \{1, 2, \dots, k-1\} &&\text{(blocks are used in order)}\\
    &x_{vi}, d_{vi} \quad \forall v \in V, i \in \{1, 2, \dots, k\} &&\text{(binary variables)}\\
\end{align*}


\section{Planned Contributions}
\begin{itemize}[leftmargin=*, noitemsep]
    \item \textbf{Exact Modeling}: We provide the first known exact integer programming model for the dominator coloring problem.
    \item \textbf{Model Strengthening}: We aim to derive \textit{valid inequalities} to strengthen the above formulation.
    \item \textbf{Computational Comparisons}: We will perform computational experiments on different graph sizes (small, medium, large) to:
    \begin{itemize}[noitemsep]
        \item Compare the \textbf{original model} and \textbf{strengthened model}.
        \item Analyze improvements in \textbf{computational time} and \textbf{solver performance}.
    \end{itemize}
    \item \textbf{New Formulation Development}: Optionally, we may propose a \textbf{new mathematical model} based on alternative representations (e.g., alternative graph properties) and compare it against the original one in terms of both \textbf{formulation strength} and \textbf{computational efficiency}.
\end{itemize}

\section{Conclusion}
The Dominator Coloring Problem is a challenging area within graph optimization. By formulating the problem as an integer program and analyzing its properties through computational experiments, we aim to provide a new and valuable perspective for solving and understanding dominator colorings.



\bibliographystyle{plain}
\begin{thebibliography}{9}

\bibitem{dominator_partitions}
S. M. Hedetniemi, S. T. Hedetniemi, R. Laskar, A. A. McRae, C. K. Wallis,
\textit{Dominator Partitions of Graphs}, 2008.

\end{thebibliography}

\end{document}
